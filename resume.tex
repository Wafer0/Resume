\documentclass[10pt, letterpaper]{article}

% Packages:
\usepackage[
    ignoreheadfoot, % set margins without considering header and footer
    top=1.0cm, bottom=1.0cm, left=1.5cm, right=1.5cm, % Tighter margins to ensure one page
    footskip=1.0 cm, % seperation between body and footer
    % showframe % for debugging
]{geometry} % for adjusting page geometry
\usepackage{titlesec} % for customizing section titles
\usepackage{tabularx} % for making tables with fixed width columns
\usepackage{array} % tabularx requires this
\usepackage[dvipsnames]{xcolor} % for coloring text
\definecolor{primaryColor}{RGB}{0, 0, 0} % define primary color
\usepackage{enumitem} % for customizing lists
\usepackage{fontawesome5} % for using icons
\usepackage{amsmath} % for math
\usepackage[
    pdftitle={Andreas Tzitzikas's CV},
    pdfauthor={Andreas Tzitzikas},
    pdfcreator={LaTeX with RenderCV},
    colorlinks=true,
    urlcolor=primaryColor
]{hyperref} % for links, metadata and bookmarks
\usepackage[pscoord]{eso-pic} % for floating text on the page
\usepackage{calc} % for calculating lengths
\usepackage{bookmark} % for bookmarks
\usepackage{lastpage} % for getting the total number of pages
\usepackage{changepage} % for one column entries (adjustwidth environment)
\usepackage{paracol} % for two and three column entries
\usepackage{ifthen} % for conditional statements
\usepackage{needspace} % for avoiding page brake right after the section title
\usepackage{iftex} % check if engine is pdflatex, xetex or luatex
\usepackage{datetime} % For \today

% Ensure that generate pdf is machine readable/ATS parsable:
\ifPDFTeX
    \input{glyphtounicode}
    \pdfgentounicode=1
    \usepackage[T1]{fontenc}
    \usepackage[utf8]{inputenc}
    \usepackage{lmodern}
\fi

\usepackage{charter}

% Some settings:
\raggedright
\AtBeginEnvironment{adjustwidth}{\partopsep0pt} % remove space before adjustwidth environment
\pagestyle{empty} % no header or footer
\setcounter{secnumdepth}{0} % no section numbering
\setlength{\parindent}{0pt} % no indentation
\setlength{\topskip}{0pt} % no top skip
\setlength{\columnsep}{0.2cm} % set column seperation
\pagenumbering{gobble} % no page numbering

\titleformat{\section}{\needspace{4\baselineskip}\bfseries\large}{}{0pt}{}[\vspace{1pt}\titlerule]

\titlespacing{\section}{
    % left space:
    -1pt
}{
    % top space:
    0.3 cm
}{
    % bottom space:
    0.2 cm
} % section title spacing

\renewcommand\labelitemi{$\vcenter{\hbox{\small$\bullet$}}$} % custom bullet points
\newenvironment{highlights}{
    \begin{itemize}[
        topsep=0.10 cm,
        parsep=0.10 cm,
        partopsep=0pt,
        itemsep=0pt,
        leftmargin=0 cm + 10pt
    ]
}{
    \end{itemize}
} % new environment for highlights

\newenvironment{onecolentry}{
    \begin{adjustwidth}{
        0 cm + 0.00001 cm
    }{
        0 cm + 0.00001 cm
    }
}{
    \end{adjustwidth}
} % new environment for one column entries

\newenvironment{twocolentry}[2][]{
    \onecolentry
    \def\secondColumn{#2}
    \setcolumnwidth{\fill, 4.5 cm}
    \begin{paracol}{2}
}{
    \switchcolumn \raggedleft \secondColumn
    \end{paracol}
    \endonecolentry
} % new environment for two column entries

\newenvironment{header}{
    \setlength{\topsep}{0pt}\par\kern\topsep\centering\linespread{1.5}
}{
    \par\kern\topsep
} % new environment for the header

% save the original href command in a new command:
\let\hrefWithoutArrow\href

\begin{document}
    \newcommand{\AND}{\unskip
        \cleaders\copy\ANDbox\hskip\wd\ANDbox
        \ignorespaces
    }
    \newsavebox\ANDbox
    \sbox\ANDbox{$|$}

    \begin{header}
        \fontsize{18 pt}{18 pt}\selectfont Andreas Tzitzikas

        \vspace{5 pt}

        \normalsize
        \mbox{College Park, MD}%
        \kern 5.0 pt%
        \AND%
        \kern 5.0 pt%
        \mbox{\hrefWithoutArrow{mailto:atzi@tutamail.com}{andreas.tz.work@gmail.com}}%
        \kern 5.0 pt%
        \AND%
        \kern 5.0 pt%
        \mbox{\hrefWithoutArrow{tel:410-920-5319}{410-920-5319}}%
        \kern 5.0 pt%
        \AND%
        \kern 5.0 pt%
        \mbox{\hrefWithoutArrow{https://linkedin.com/in/andreas-tzitzikas}{linkedin.com/in/andreas-tzitzikas}}%
        \kern 5.0 pt%
        \AND%
        \kern 5.0 pt%
        \mbox{\hrefWithoutArrow{https://github.com/Wafer0}{github.com/Wafer0}}%
    \end{header}

    \vspace{5 pt - 0.3 cm}

    \section{Education}

        \begin{twocolentry}{
            Expected May 2026
        }
            \textbf{University of Maryland}, College Park, MD
        \end{twocolentry}
        \begin{twocolentry}{
            \textbf{GPA: 3.88}
        }
            Bachelor of Science in Computer Engineering
        \end{twocolentry}

    \section{Experience}

        \begin{twocolentry}{
            Jun 2025 – Aug 2025
        }
            \textbf{Embedded Systems Intern}, Alchemity -- College Park, MD
        \end{twocolentry}
        \vspace{0.10 cm}
        \begin{onecolentry}
            \begin{highlights}
                \item Developed desktop and embedded software tools to support process automation and equipment control in a research setting, using Rust, Electron.js, and Python.
                \item Designed real-time firmware for STM32-based microcontrollers with support for non-blocking motor and relay control, SPI peripherals, and user interface elements.
                \item Built end-to-end control interfaces including desktop GUIs, embedded display drivers, and command-line tools to streamline operator workflows.
                \item Contributed to the design and deployment of robust, fault-tolerant embedded control systems to maintain operational safety and uptime in laboratory environments.
            \end{highlights}
        \end{onecolentry}

        \vspace{0.2 cm}

        \begin{twocolentry}{
            Jun 2024 – Present
        }
            \textbf{Technical Coordinator}, Electronics Prototyping Lab, Terrapin Works -- College Park, MD \end{twocolentry}
        \vspace{0.10 cm}
        \begin{onecolentry}
            \begin{highlights}
                \item Partnered with 20+ students and faculty per semester to provide end-to-end support on more than than a dozen cuPCBs - from - from schematic design toassembly - whiley - while performing all maintenance on LPKF manufacturing tools to ensure 90\% uptime.
                \item Led training for new employees and expanded campus outreach to increase awareness of PCB services, while assisting researchers in the Instructional Electronics Lab with diagnosing and building devices.
            \end{highlights}
        \end{onecolentry}
        
    \section{Projects}
    
        \begin{twocolentry}{
             July 2025 - September 2025 } \textbf{ 5-Stage Pipelined RISC-V CPU}
        \end{twocolentry}
        \vspace{0.10 cm}
        \begin{onecolentry}
            \begin{highlights}
                \item Designed and implemented a complete 5-stage pipelined CPU architecture (Fetch, Decode, Execute, Memory, Write-Back) supporting the RISC-V ISA, including hazard detection and forwarding units.
                \item Developed and verified the design using SystemVerilog, performing synthesis and implementation on a Xilinx FPGA and validating against a suite of assembly-level test programs.
            \end{highlights}
        \end{onecolentry}

        \vspace{0.2 cm}

        \begin{twocolentry}{
            Feb 2025 – Apr 2025
        }
            \textbf{Low-Power 6-bit Dadda Multiplier in CMOS}
        \end{twocolentry}
        \vspace{0.10 cm}
        \begin{onecolentry}
            \begin{highlights}
                \item Designed and implemented a high-performance 6-bit Dadda multiplier optimized for low power and minimal propagation delay, targeting efficient deep neural network acceleration.
                \item Manual optimization of the transistor level including gate sizing, strategic transistor placement, and layout refinement to reduce area and power consumption.
                \item Achieved significant improvements in performance by balancing speed, power, and area (PPA) trade-offs, with final metrics verified through Cadence Specter simulations.
            \end{highlights}
        \end{onecolentry}

        \vspace{0.2 cm}

        \begin{twocolentry}{
            Jan 2025 – May 2025
        }
            \textbf{STM32G4 Bare-Metal Peripheral Integration}
        \end{twocolentry}
        \vspace{0.10 cm}
        \begin{onecolentry}
            \begin{highlights}
                \item Designed and implemented firmware modules in the ARM Assembly for the STM32G491RE microcontroller, enabling complete control over various hardware peripherals.
                \item Performed direct register-level programming for peripherals, including GPIO, DAC, ADC, Timers (PWM / PPM / OPM) and SPI for external W25QXX flash memory.
                \item Developed and managed shared EXTI interrupt routines for GPIO, DAC, ADC, and TIM2.
            \end{highlights}
        \end{onecolentry}

    \section{Skills}
        \begin{onecolentry}
            \textbf{Languages \& HDLs:} C/C++, Python, Rust, SystemVerilog, Verilog, ARM Assembly \\
            \textbf{Computer Architecture:} RISC-V \& ARM ISAs, Pipelined Datapaths, Hazard Control, CMOS VLSI Design, RTL Design \\
            \textbf{EDA Tools \& Methodologies:} Xilinx Vivado, ModelSim/QuestaSim, Cadence Spectre, FPGA Synthesis \& Implementation, Testbench Development, PPA Optimization \\
            \textbf{Embedded Systems:} STM32 Bare-Metal Firmware, Real-Time Control, PCB Design, SPI, I2C, UART, GPIO, ADC/DAC
        \end{onecolentry}

\end{document}
